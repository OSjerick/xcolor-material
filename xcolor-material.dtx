% \iffalse meta-comment
%
% Copyright (C) 2016 by Jerick Órdenes Sepúlveda <os.jerick+xcolormaterial at gmail.com>
% -------------------------------------------------------
% 
% This file may be distributed and/or modified under the
% conditions of the LaTeX Project Public License, either version 1.3
% of this license or (at your option) any later version.
% The latest version of this license is in:
%
%    http://www.latex-project.org/lppl.txt
%
% and version 1.3 or later is part of all distributions of LaTeX 
% version 2005/12/01 or later.
%
% \fi
%
% \iffalse
%<*driver>
\ProvidesFile{xcolor-material.dtx}
%</driver>
%<package>\NeedsTeXFormat{LaTeX2e}[2005/12/01]
%<package>\ProvidesPackage{xcolor-material}
%<package>[2016/10/11 v0.1 A package for accessing Google Material colors using the xcolor package]
%
%<*driver>
\PassOptionsToClass{a4paper}{article}
\documentclass{ltxdoc}
\EnableCrossrefs
\CodelineIndex
\RecordChanges

\usepackage{lmodern}
\usepackage[utf8]{inputenc}
\usepackage[T1]{fontenc}
\PassOptionsToPackage{table}{xcolor}
\usepackage{xcolor-material}[2016/10/11]
\newcolumntype{L}[1]{>{\raggedright\let\newline\\\arraybackslash\hspace{0pt}}m{#1}}
\newcolumntype{C}[1]{>{\centering\let\newline\\\arraybackslash\hspace{0pt}}m{#1}}
\newcolumntype{R}[1]{>{\raggedleft\let\newline\\\arraybackslash\hspace{0pt}}m{#1}}
\usepackage{tikz}
\usepackage[stretch=10]{microtype}
\usepackage{lstdoc}
\makeatletter
\def\@part[#1]#2{\ifhyper\phantomsection\fi
    \addcontentsline{toc}{part}{\sffamily #1}%
    {\parindent\z@ \raggedright \interlinepenalty\@M
        \sffamily \huge \bfseries \color{MaterialBlue700} #2\markboth{}{}\par}%
    \nobreak\vskip 3ex\@afterheading}
\makeatother
\hypersetup{linkcolor=MaterialRed600, urlcolor=MaterialPink600}
\usepackage{cleveref}

\newcommand*{\pkg}[1]{\textsf{#1}}
\newcommand*{\opt}[1]{\texttt{#1}}
\newcommand\xcolorpkg{\pkg{xcolor}}
\newcommand\kvopkg{\pkg{kvoptions}}
\newcommand\xmpkg{\pkg{xcolor-material}}
\newcommand\google{{\sffamily{\color{GoogleBlue}G}{\color{GoogleRed}o}%
  {\color{GoogleYellow}o}{\color{GoogleBlue}g}{\color{GoogleGreen}l}{\color{GoogleRed}e}}}
\newcommand\tikzpkg{PGF/Ti\textit{k}Z}

\makeindex
\begin{document}
    \DocInput{xcolor-material.dtx}
    \PrintChanges
    \PrintIndex
\end{document}
%</driver>
% \fi
%
% \CheckSum{567}
%
% \CharacterTable
%  {Upper-case    \A\B\C\D\E\F\G\H\I\J\K\L\M\N\O\P\Q\R\S\T\U\V\W\X\Y\Z
%   Lower-case    \a\b\c\d\e\f\g\h\i\j\k\l\m\n\o\p\q\r\s\t\u\v\w\x\y\z
%   Digits        \0\1\2\3\4\5\6\7\8\9
%   Exclamation   \!     Double quote  \"     Hash (number) \#
%   Dollar        \$     Percent       \%     Ampersand     \&
%   Acute accent  \'     Left paren    \(     Right paren   \)
%   Asterisk      \*     Plus          \+     Comma         \,
%   Minus         \-     Point         \.     Solidus       \/
%   Colon         \:     Semicolon     \;     Less than     \<
%   Equals        \=     Greater than  \>     Question mark \?
%   Commercial at \@     Left bracket  \[     Backslash     \\
%   Right bracket \]     Circumflex    \^     Underscore    \_
%   Grave accent  \`     Left brace    \{     Vertical bar  \|
%   Right brace   \}     Tilde         \~}
%
%
% \changes{v0.1}{2016/10/11}{Initial package version}
%
% \GetFileInfo{xcolor-material.sty}
%
% \DoNotIndex{\newcommand,\begin,\end,\bfseries,\sffamily,\ttfamily,\texttt,
% \textbf,\emph,\auxcolor,\baselineskip,\color,\XCM@CMYKModel,\color,\colorval,
% \colorbox,\parbox,\CurrentOption,\convertcolorspec,\DeclareDefaultOption,
% \DeclareStringOption,\def,\edef,\definecolor,\else,\fi,\empty,\expandafter,
% \#,\@ifpackageloaded,\extractcolorspecs,\extractedcolor,\first,\footnotesize,
% \foreach,\gdef,\hspace,\i,\ifx,\model,\node,\OptionNotUsed,\PackageError,
% \pgfkeys,\printcolorvalue,\ProcessKeyvalOptions,\raggedright,
% \selectcolormodel,\textwidth,\tikzset,\usetikzlibrary,\variant,\vfill,
% \vspace,\@ifnextchar}
% 
%
%\newbox\abstractbox
%\setbox\abstractbox=\vbox{
% \begin{abstract}
% The \xmpkg\ package is built on top of the great \xcolorpkg\ package. It 
% exists for providing a useful definition of the beautiful \google\ Material 
% Color Palette, available from 
% \url{https://material.google.com/style/color.html}, for its use in document 
% writing with \LaTeX\ and Friends.
% \end{abstract}}
% \title{The \xmpkg\ package}
% \author{Jerick Órdenes Sepúlveda \\
% <\href{mailto:os.jerick+xcolor-material@gmail.com}%
% {os.jerick+xcolormaterial(at)gmail.com}>}
% \date{\filedate~~ Version \fileversion\ \box\abstractbox}
%
% \csname @twocolumntrue\endcsname
% \maketitle\thispagestyle{empty}
% \csname @starttoc\endcsname{toc}
% \onecolumn
%
% \part{Usage}
%
% \section{Installation}
%
% \subsection{Required Packages}
% The \xmpkg\ package requires relatively up-to-date versions of the packages: 
% \kvopkg\footnote{The \kvopkg\ package is available from
% \url{https://www.ctan.org/pkg/kvoptions}} and \xcolorpkg\footnote{The 
% \xcolorpkg\ package is available from \url{https://www.ctan.org/pkg/xcolor}}, 
% which are distributed with any popular \TeX\ distribution. Both packages are 
% loaded without any options.
%
% \subsection{Installing \xmpkg}
% If the package is released on \href{http://www.ctan.org}{CTAN}, the users
% should be able to install it through the \TeX\ distribution package manager
% available on their PC. Nevertheless, if you need to install the package 
% manually, you should run
% ^^A
% \begin{verbatim}
%   latex xcolor-material.ins\end{verbatim}
% ^^A
% and copy the generated file |xcolor-material.sty| to a path where \LaTeX\ can 
% find it. To generate the documentation you should run the following 
% command twice.
% ^^A
% \begin{verbatim}
%   pdflatex xcolor-material.dtx\end{verbatim}
%
% \section{Loading \xmpkg}
% Simply add the package declaration somewhere in the document preamble.
% ^^A
% \begin{verbatim}
%   \usepackage{xcolor-material}\end{verbatim}
%
% If you want to load the \kvopkg\ and \xcolorpkg\ packages with some 
% options, you can specify those options using the |\PassOptionsToPackage| 
% command \emph{before} loading \xmpkg, or by loading those packages with their 
% options \emph{before} loading the \xmpkg\ package.
%
% \subsection{Package Options}
% The \xmpkg\ currently offers only one option,
% \begin{description}
%   \item[\opt{prefix}|=|\meta{prefix}] Defines the prefix for all color names
%   defined in the Google Material Color Palette; this is used for namespacing
%   purposes. The default prefix is ``|Material|''.
% \end{description}
% The main purpose of this option is to avoid conflicts with other existent 
% color definitions. The |prefix| option allows custom prefix definition. For
% instance, if you want to use the word `|myprefix|' as the color prefix, 
% then you must load the \xmpkg\ package with
% ^^A
% \begin{verbatim}
%   \usepackage[prefix=myprefix]{xcolor-material}\end{verbatim}
%
% \section{Using \xmpkg}
% The \xmpkg\ provides all colors defined in the Google Material Color 
% Palette, available at \url{https://material.google.com/style/color.html}, at 
% the global scope of the document. This package relies on the \xcolorpkg\ 
% package, and you should refer to its documentation for more details 
% about using colors inside the document.
%
% \subsection{Color Names}
% The Google Material Color Palette consists of another 19 color palettes (each 
% for one base color) and two colors not listed into a palette (|White| and 
% |Black|), together forming a list of 256 colors. Each individual palette 
% consists of 14 color variations of the base color, except for |Brown|, 
% |Grey|, and |Blue Grey| with 10. Every color variation has a number from the 
% list \texttt{\{50, 100, 200, 300, 400, 500, 600, 700, 800, 900\}}  for the 
% first ten colors or \emph{primary palette}, and from the list \texttt{\{A100, 
% A200, A300, A400\}} for the last four or \emph{secondary palette}. The names 
% of the base colors and their variations are listed in table 
% \ref{tab:colornames}. For a complete list of these palettes go to the section 
% \ref{sec:colorpalettes}.
%
% \begin{table}[b!]
%   \centering
%   \caption{Color names from the Google Material Color Palette and their 
%   variation palettes.} \label{tab:colornames}
%   \begin{tabular}{lc}
%      \hline
%      \textbf{Base Color} & \textbf{Palettes}\\
%      \hline
%      \textcolor{MaterialRed}{|Red|} & primary, secondary\\
%      \textcolor{MaterialPink}{|Pink|} & primary, secondary\\
%      \textcolor{MaterialPurple}{|Purple|} & primary, secondary\\
%      \textcolor{MaterialDeepPurple}{|Deep Purple|} & primary, secondary\\
%      \textcolor{MaterialIndigo}{|Indigo|} & primary, secondary\\
%      \textcolor{MaterialBlue}{|Blue|} & primary, secondary\\
%      \textcolor{MaterialLightBlue}{|Light Blue|} & primary, secondary\\
%      \textcolor{MaterialCyan}{|Cyan|} & primary, secondary\\
%      \textcolor{MaterialTeal}{|Teal|} & primary, secondary\\
%      \textcolor{MaterialGreen}{|Green|} & primary, secondary\\
%      \textcolor{MaterialLightGreen}{|Light Green|} & primary, secondary\\
%      \textcolor{MaterialLime}{|Lime|} & primary, secondary\\
%      \textcolor{MaterialYellow}{|Yellow|} & primary, secondary\\
%      \textcolor{MaterialAmber}{|Amber|} & primary, secondary\\
%      \textcolor{MaterialOrange}{|Orange|} & primary, secondary\\
%      \textcolor{MaterialDeepOrange}{|Deep Orange|} & primary, secondary\\
%      \textcolor{MaterialBrown}{|Brown|} & secondary\\
%      \textcolor{MaterialGrey}{|Grey|} & secondary\\
%      \textcolor{MaterialBlueGrey}{|Blue Grey|} & secondary\\
%      \textcolor{MaterialBlack}{|Black|} & ---\\
%      \rowcolor{MaterialBlack}\textcolor{MaterialWhite}{|White|} & 
% \textcolor{MaterialWhite}{---}\\
%      \hline
%    \end{tabular}
%\end{table}
%
% The name under which the \xmpkg\ package defines each Google Material color 
% has the form `\meta{prefix}\meta{base name}\meta{variation}', where
% \begin{itemize}
%   \item \meta{prefix} corresponds to the prefix set via the package option
%   |prefix|,
%   \item \meta{base name} is the official name of the base color (see table
%   \ref{tab:colornames}), and
%   \item \meta{variation} is the id number of the color variation, e.g., |50| 
% or |A100|.
% \end{itemize}
% For instance, by default (if a custom prefix is not defined), the |Blue 600| 
% color of the palette will be available in the document global scope as 
% ``|MaterialBlue600|''. Also, all base colors are representative of primary 
% colors and are available at the |500| variation or at the base color name, 
% e.g., |MaterialBlue| or |MaterialBlue500|.
%
% \subsection[Google Colors]{\google{} Colors}
% The \xmpkg\ package provides the Google Identity colors, which definitions 
% are available at the Google Design Blog from
% \url{https://design.google.com/articles/evolving-the-google-identity}. 
% The name under which these colors are defined has the form 
% `Google\meta{name}', where \meta{name} is a color name among 
% \textcolor{GoogleBlue}{|Blue|}, \textcolor{GoogleGreen}{|Green|}, 
% \textcolor{GoogleYellow}{|Yellow|}, and \textcolor{GoogleRed}{|Red|}. For 
% instance, if you want to use the Google Blue color you have to call for it 
% by using `|GoogleBlue|'.\\
%
% \noindent
% \colorsample{GoogleBlue}
% \colorsample{GoogleGreen}
% \colorsample{GoogleYellow}
% \colorsample{GoogleRed}\\
%
% \subsection{Useful macros}
% The \xmpkg\ provides some useful macros for printing color values, samples, 
% and palettes. Some of them maybe rely on other LaTeX packages that are 
% not loaded by default.
%
% \noindent
% \DescribeMacro{\printcolorvalue}
% \oarg{color model}\marg{color name}
%
% This macro can be used to print any color value. It has one mandatory 
% argument \meta{color name}, which specifies the color name which you want 
% to know its value. And one optional argument, \meta{color model} that 
% corresponds to the color model in which the value will be presented: |HTML|, 
% |RGB| and |cmyk| (|HTML| by default).
% \begin{table}[h!]
%   \centering
%   \begin{tabular}{lll}
%     |\printcolorvalue{MaterialBlue}| & & \textcolor{MaterialIndigo}{\printcolorvalue{MaterialBlue}}\\
%     |\printcolorvalue[RGB]{MaterialGrey}| & & \textcolor{MaterialIndigo}{\printcolorvalue[RGB]{MaterialGrey}}\\
%     |\printcolorvalue[cmyk]{orange}| & & \textcolor{MaterialIndigo}{\printcolorvalue[cmyk]{orange}}\\
%   \end{tabular}
%\end{table}
%
% \noindent
% \DescribeMacro{\colorsample}
% \oarg{color model}\oarg{text color}\oarg{box width}\marg{color name}\oarg{title}
%
% This macro allows you to draw color samples using a simple colorbox with
% text. It has one mandatory argument 
% \begin{itemize}
%   \item \meta{color name} which is the name of the color to be drawn,
% \end{itemize}
% and four optional arguments: 
% \begin{itemize}
%   \item \meta{color model} represents the color model in which the color will 
%   be presented (by default |HTML|),
%   \item \meta{text color} which specifies the color used for printing the 
%   text (by default |white|),
%   \item \meta{box width} specifies the box width (by default is |6.5em|), if 
%   it is \emph{empty} the default is passed, \item and \meta{title} which is 
%   the title of the color sample, if this argument is not passed the color 
%   name is taken.
% \end{itemize}
%
% To generate samples of the \google\ Colors showed above we can type 
% \begin{verbatim}
%   \colorsample{GoogleBlue} \colorsample{GoogleGreen}
%   \colorsample{GoogleYellow} \colorsample{GoogleRed}\end{verbatim}
%
% Other samples:
% ^^A 
% \begin{verbatim}
%   \colorsample[HTML][][black]{MaterialAmber50}
%   \colorsample{MaterialAmber}
%   \colorsample[RGB]{MaterialAmber600}
%   \colorsample{MaterialAmberA700}[Amber\\A700]\end{verbatim}
%
% \noindent
% \colorsample[HTML][][black]{MaterialAmber50}
% \colorsample{MaterialAmber}
% \colorsample[RGB]{MaterialAmber600}
% \colorsample{MaterialAmberA700}[Amber\\A700]
%
% \begin{verbatim}
%   \colorsample[cmyk][13em]{MaterialBlueGrey500}
%   \colorsample{purple}[\texttt{xcolor}\\ Purple]
%   \colorsample[RGB]{teal}[]
% \end{verbatim}
%
% \noindent
% \colorsample[cmyk][13em]{MaterialBlueGrey500}
% \colorsample{purple}[\texttt{xcolor}\\ Purple]
% \colorsample[RGB]{teal}[]\\
%
% \noindent
% \DescribeMacro{\colorpalette}
% \oarg{key-value arguments}\marg{color name}
%
% This macro is available only if the \tikzpkg\ package is loaded. It provides 
% a simple way to draw the Google Material Color Palettes in the same way as we 
% can find them into the \href{https://material.google.com/style/color.html}{Google 
% Style Blog}. It has one mandatory argument \meta{color name}, which specifies 
% the palette base color that will be used to draw its variation palette. 
% Optional arguments are provided in the form of key-value arguments and key 
% styles. A complete list of key-value arguments and styles is available at 
% table \ref{tab:keyvalue} and \ref{tab:styles} respectively. The \xmpkg\ 
% package relies on the \textsf{pgfkeys} package to provide key-value 
% arguments, for deeper knowledge about how to use key-value arguments look at 
% package documentation\footnote{The \textsf{pgfkeys} package is available from 
% \url{https://www.ctan.org/pkg/pgfkeys}.}.
%
% \begin{table}[t!]
%   \centering\small
%   \caption{Key-value arguments for the \texttt{\textbackslash colorpalette} 
%   macro.} \label{tab:keyvalue}
%   \rowcolors{2}{lightgray!20}{white}
%   \begin{tabular}{L{.3\textwidth}C{.25\textwidth}L{.35\textwidth}}
%     \hline
%     \multicolumn{1}{c}{\textbf{Key}}  & \multicolumn{1}{c}{\textbf{Default}}
% &   \multicolumn{1}{c}{\textbf{Description}}\\
%     \hline
%     |width| & .45 $\times$ |\textwidth| & Palette width\\
%     |height| & 2 $\times$ |\baselineskip| & Variation colorbox height\\
%     |shape| & rectangle & Variation colorbox shape\\
%     |title| & \meta{color name} & Palette title\\
%     |title text color| & white & Title colorbox text color\\
%     |init text color| & black & Initial variation text color\\
%     |altern text color| & white & Alternative variation text color\\
%     |variation font| & sffamily & Variation title font\\
%     |value font| & ttfamily & Variation value font\\
%     |title font| & sffamily and bfseries & Palette title font\\
%     |colorbox sep| & 0pt & Separation between 
% variation colorboxes\\
%     |first colorbox sep| & .2 $\times$ |\baselineskip| & Separation between 
% title colorbox and first palette colorbox\\
%     |title colorbox height| & 6 $\times$ |\baselineskip| & Height of the 
% title colorbox\\
%     |primary palette| & \{50,100,200,\ldots,900\} & List of primary palette 
% variations \\
%     |secondary palette| & \{A100,A200,A400,A700\} & List of secondary palette 
% variations\\
%     |percent char| & \{\} & Character for color mixtures (usually |!|), this % is useful when the |basic| style is selected\\
%     |primary variation| & 500 & Primary variation or palette's base color\\
%     |primary toggle at| & 1000 & Color variation at which alternates between 
% |init text color| and |altern text color| for the primary palette\\
%     |secondary toggle at| & 1000 & Color variation at which alternates 
% between |init text color| and |altern text color| for the secondary palette\\
%     |color model| & HTML & Color Model to be used for printing the palette\\
%     \hline
%   \end{tabular}
% \end{table}
%
% \begin{table}[t!]
%   \centering\small
%   \caption{Style arguments for the \texttt{\textbackslash colorpalette} 
%   macro.} \label{tab:styles}
%   \rowcolors{2}{lightgray!20}{white}
%   \begin{tabular}{C{.2\textwidth}L{.5\textwidth}}
%     \hline
%     \multicolumn{1}{c}{\textbf{Style}}  & 
% \multicolumn{1}{c}{\textbf{Description}}\\
%     \hline
%     |both| & Settings for printing both primary and secondary palettes 
% (Default)\\
%     |primary| & For printing primary palette only  \\
%     |basic| & For printing variations from 10\,\% to 100\,\% of a base color\\
%     \hline
%   \end{tabular}
% \end{table}
%
% A printed palette is a stack of colorboxes that comprises three main 
% sections, one on top or |title colorbox| that is used for introducing the 
% palette with its primary variation, a |primary palette| in the middle and a 
% |secondary palette| at the bottom. With this macro, you can print out 
% all the palettes of the Google Material Colors, and with some configurations 
% you can print any color palette that you want in a fancy way as the Material 
% Palettes.
%
% For instance, the following commands produce a color palette for the 
% variations of the \xcolorpkg\ predefined |yellow| color between 10\,\% and 
% 100\,\%, and a custom palette by tricking the |\colorpalette| macro.
% \begin{verbatim}
%   \colorpalette[basic, title text color=black]{yellow}
%   \colorpalette[primary palette={yellow,blue,red,gray,black,purple}, 
%     secondary palette={yellow!20,blue!30,red!20,gray!40}, primary 
%     variation=orange, primary toggle at=blue]{}
%\end{verbatim}
%
% \noindent
% \colorpalette[basic, title text color=black]{yellow}\hfill
% \colorpalette[primary palette={yellow,blue,red,gray,black,purple}, 
% secondary palette={yellow!20,blue!30,red!20,gray!40}, primary 
% variation=orange, primary toggle at=blue, title=Predefined Orange]{}
%
% \subsection{Color Palettes}\label{sec:colorpalettes}
% Here I present all the Google Material Palettes in the same way as they 
% appear at the Google Design blog using the |\colorpalette| command described
% above.
%
% \begin{verbatim}
%   \colorpalette[primary toggle at=400, secondary toggle at=A200, secondary 
%     palette={A100,A200,A400,A700}]{MaterialRed}
%   \colorpalette[primary toggle at=400, secondary toggle at=A200, secondary 
%     palette={A100,A200,A400,A700}]{MaterialPink}
% \end{verbatim}
%
% \noindent
% \colorpalette[primary toggle at=400, secondary toggle at=A200, secondary 
% palette={A100,A200,A400,A700}]{MaterialRed}\hfill
% \colorpalette[primary toggle at=400, secondary toggle at=A200, secondary 
% palette={A100,A200,A400,A700}]{MaterialPink}
% \clearpage
%
% \begin{verbatim}
%   \colorpalette[primary toggle at=300, secondary toggle at=A200, secondary 
%     palette={A100,A200,A400,A700}]{MaterialPurple}
%   \colorpalette[primary toggle at=300, secondary toggle at=A200, secondary 
%     palette={A100,A200,A400,A700}]{MaterialDeepPurple}
% \end{verbatim}
%
% \noindent
% \colorpalette[primary toggle at=300, secondary toggle at=A200, secondary 
% palette={A100,A200,A400,A700}]{MaterialPurple}\hfill
% \colorpalette[primary toggle at=300, secondary toggle at=A200, secondary 
%  palette={A100,A200,A400,A700}]{MaterialDeepOrange}
% \clearpage
%
% \begin{verbatim}
%  \colorpalette[primary toggle at=500, secondary toggle at=A200, secondary 
%    palette={A100,A200,A400,A700}]{MaterialBlue}
%  \colorpalette[primary toggle at=300, secondary toggle at=A200, secondary 
%    palette={A100,A200,A400,A700}]{MaterialIndigo}
% \end{verbatim}
%
% \noindent
% \colorpalette[primary toggle at=500, secondary toggle at=A200, secondary 
% palette={A100,A200,A400,A700}]{MaterialBlue}\hfill
% \colorpalette[primary toggle at=300, secondary toggle at=A200, secondary 
% palette={A100,A200,A400,A700}]{MaterialIndigo}
% \clearpage
%
% \begin{verbatim}
%   \colorpalette[primary toggle at=600, secondary toggle at=A700, title text 
%     color=black]{MaterialLightBlue}
%   \colorpalette[primary toggle at=700, title text color=black]{MaterialCyan}
% \end{verbatim}
%
% \noindent
% \colorpalette[primary toggle at=600, secondary toggle at=A700, title text 
% color=black]{MaterialLightBlue}\hfill
% \colorpalette[primary toggle at=700, title text color=black]{MaterialCyan}
% \clearpage
%
% \begin{verbatim}
%   \colorpalette[primary toggle at=500, secondary 
%     palette={A100,A200,A400,A700}]{MaterialTeal}
%   \colorpalette[primary toggle at=600, title text color=black]{MaterialGreen}
% \end{verbatim}
%
% \noindent
% \colorpalette[primary toggle at=500, secondary 
% palette={A100,A200,A400,A700}]{MaterialTeal}\hfill
% \colorpalette[primary toggle at=600, title text color=black]{MaterialGreen}
% \clearpage
%
% \begin{verbatim}
%   \colorpalette[primary toggle at=700, title text 
%     color=black]{MaterialLightGreen}
%   \colorpalette[primary toggle at=900, title text color=black]{MaterialLime}
% \end{verbatim}
%
% \noindent
% \colorpalette[primary toggle at=700, title text 
% color=black]{MaterialLightGreen}\hfill
% \colorpalette[primary toggle at=900, title text color=black]{MaterialLime}
% \clearpage
%
% \begin{verbatim}
%   \colorpalette[title text color=black]{MaterialYellow}
%   \colorpalette[title text color=black]{MaterialAmber}
% \end{verbatim}
%
% \noindent
% \colorpalette[title text color=black]{MaterialYellow}\hfill
% \colorpalette[title text color=black]{MaterialAmber}
% \clearpage
%
% \begin{verbatim}
%   \colorpalette[primary toggle at=800, title text color=black]{MaterialOrange}
%   \colorpalette[primary toggle at=500, secondary toggle 
%     at=A400]{MaterialDeepOrange}
% \end{verbatim}
%
% \noindent
% \colorpalette[primary toggle at=800, title text 
% color=black]{MaterialOrange}\hfill
% \colorpalette[primary toggle at=500, secondary toggle 
% at=A400]{MaterialDeepOrange}
% \clearpage
%
% \begin{verbatim}
%   \colorpalette[primary, primary toggle at=300]{MaterialBrown}
%   \colorpalette[primary, primary toggle at=600]{MaterialGrey}
% \end{verbatim}
%
% \noindent
% \colorpalette[primary, primary toggle at=300]{MaterialBrown}\hfill
% \colorpalette[primary, primary toggle at=600]{MaterialGrey}
% \clearpage
%
% \begin{verbatim}
%   \colorpalette[primary, primary toggle at=400]{MaterialBlueGrey}
% \end{verbatim}
%
% \noindent
% \colorpalette[primary, primary toggle at=400]{MaterialBlueGrey}
%
% \section{Bug reports}
% The development version of this package is available at 
% \url{https://example.com}. If you find bugs, have feature suggestions or 
% think of ways in which the \xmpkg\ package and its documentation could be 
% improved, please open a ticket in the GitHub repository's issue tracker.
%
% \section{Acknowledgements}
% This package was inspired by the \textsf{xcolor-solarized} package, available
% from \url{https://www.ctan.org/pkg/xcolor-solarized}, then I have to thank 
% Julien Cretel its creator. Also, thanks to \google\ for such amazing color 
% palette.
%
% \StopEventually{That's all!}
% \clearpage
% \part{Implementation}
%
% I have used the word `|XCM|' followed by an `|@|' character for namespacing
% purposes into the \xmpkg\ package source.
%
% \section{Required Packages}
% The \xmpkg\ requires the \xcolorpkg\ and \kvopkg\ packages, both are 
% loaded without any option.
%
%    \begin{macrocode}
\RequirePackage{xcolor}[2007/01/21]
\RequirePackage{kvoptions}[2011/06/30]
%    \end{macrocode}
%
% The \xcolorpkg\ package is used to define and provide every single color in 
% the \google\ Material Color Palette. The \kvopkg\ package supports package 
% options in a key-value format.
%
% \section{Package Options}
% Setting up \kvopkg\ configuration.
%
%    \begin{macrocode}
\SetupKeyvalOptions{
  family=XCM,
  prefix=XCM@
}
%    \end{macrocode}
%
% Declaring string option key `|prefix|':
%
%    \begin{macrocode}
\DeclareStringOption[Material]{prefix}[Material]
\DeclareDefaultOption{
  \OptionNotUsed
  \PackageError{xcolor-material}{Unknown `\CurrentOption' option.}
}
\ProcessKeyvalOptions*
%    \end{macrocode}
%
% Note that the default value for this key is `|Material|'. Note also that this
% package checks for valid options, if an invalid option is passed, a 
% PackageError will be thrown.
%
% \section{Color Definitions}
%
% \begin{macro}{\XCM@definecolor}
% I have taken the idea from the \textsf{xcolor-solarized} package of having a 
% convenient internal macro for color definitions.
%    \begin{macrocode}
\newcommand{\XCM@definecolor}[2]
  {\expandafter\definecolor\expandafter{\XCM@prefix #1}{HTML}{#2}}
%    \end{macrocode}
% \end{macro}
%
% Using this macro definition we can define all the \google\ Material colors
% as follows. Note that for every color in the palette, the |500|
% variation is the base color default as \google\ recommends for primary colors.
%
% \subsection{Red}
%    \begin{macrocode}
\XCM@definecolor{Red50}  {FFEBEE}
\XCM@definecolor{Red100} {FFCDD2}
\XCM@definecolor{Red200} {EF9A9A}
\XCM@definecolor{Red300} {E57373}
\XCM@definecolor{Red400} {EF5350}
\XCM@definecolor{Red500} {F44336}
\XCM@definecolor{Red600} {E53935}
\XCM@definecolor{Red700} {D32F2F}
\XCM@definecolor{Red800} {C62828}
\XCM@definecolor{Red900} {B71C1C}
\XCM@definecolor{RedA100}{FF8A80}
\XCM@definecolor{RedA200}{FF5252}
\XCM@definecolor{RedA400}{FF1744}
\XCM@definecolor{RedA700}{D50000}
\definecolor{\XCM@prefix Red}{named}{\XCM@prefix Red500}
%    \end{macrocode}
%
% \subsection{Pink}
%    \begin{macrocode}
\XCM@definecolor{Pink50}  {FCE4EC}
\XCM@definecolor{Pink100} {F8BBD0}
\XCM@definecolor{Pink200} {F48FB1}
\XCM@definecolor{Pink300} {F06292}
\XCM@definecolor{Pink400} {EC407A}
\XCM@definecolor{Pink500} {E91E63}
\XCM@definecolor{Pink600} {D81B60}
\XCM@definecolor{Pink700} {C2185B}
\XCM@definecolor{Pink800} {AD1457}
\XCM@definecolor{Pink900} {880E4F}
\XCM@definecolor{PinkA100}{FF80AB}
\XCM@definecolor{PinkA200}{FF4081}
\XCM@definecolor{PinkA400}{F50057}
\XCM@definecolor{PinkA700}{C51162}
\definecolor{\XCM@prefix Pink}{named}{\XCM@prefix Pink500}
%    \end{macrocode}
%
% \subsection{Purple}
%    \begin{macrocode}
\XCM@definecolor{Purple50}  {F3E5F5}
\XCM@definecolor{Purple100} {E1BEE7}
\XCM@definecolor{Purple200} {CE93D8}
\XCM@definecolor{Purple300} {BA68C8}
\XCM@definecolor{Purple400} {AB47BC}
\XCM@definecolor{Purple500} {9C27B0}
\XCM@definecolor{Purple600} {8E24AA}
\XCM@definecolor{Purple700} {7B1FA2}
\XCM@definecolor{Purple800} {6A1B9A}
\XCM@definecolor{Purple900} {4A148C}
\XCM@definecolor{PurpleA100}{EA80FC}
\XCM@definecolor{PurpleA200}{E040FB}
\XCM@definecolor{PurpleA400}{D500F9}
\XCM@definecolor{PurpleA700}{AA00FF}
\definecolor{\XCM@prefix Purple}{named}{\XCM@prefix Purple500}
%    \end{macrocode}
%
% \subsection{Deep Purple}
%    \begin{macrocode}
\XCM@definecolor{DeepPurple50}  {EDE7F6}
\XCM@definecolor{DeepPurple100} {D1C4E9}
\XCM@definecolor{DeepPurple200} {B39DDB}
\XCM@definecolor{DeepPurple300} {9575CD}
\XCM@definecolor{DeepPurple400} {7E57C2}
\XCM@definecolor{DeepPurple500} {673AB7}
\XCM@definecolor{DeepPurple600} {5E35B1}
\XCM@definecolor{DeepPurple700} {512DA8}
\XCM@definecolor{DeepPurple800} {4527A0}
\XCM@definecolor{DeepPurple900} {311B92}
\XCM@definecolor{DeepPurpleA100}{B388FF}
\XCM@definecolor{DeepPurpleA200}{7C4DFF}
\XCM@definecolor{DeepPurpleA400}{651FFF}
\XCM@definecolor{DeepPurpleA700}{6200EA}
\definecolor{\XCM@prefix DeepPurple}{named}{\XCM@prefix DeepPurple500}
%    \end{macrocode}
%
% \subsection{Indigo}
%    \begin{macrocode}
\XCM@definecolor{Indigo50}  {E8EAF6}
\XCM@definecolor{Indigo100} {C5CAE9}
\XCM@definecolor{Indigo200} {9FA8DA}
\XCM@definecolor{Indigo300} {7986CB}
\XCM@definecolor{Indigo400} {5C6BC0}
\XCM@definecolor{Indigo500} {3F51B5}
\XCM@definecolor{Indigo600} {3949AB}
\XCM@definecolor{Indigo700} {303F9F}
\XCM@definecolor{Indigo800} {283593}
\XCM@definecolor{Indigo900} {1A237E}
\XCM@definecolor{IndigoA100}{8C9EFF}
\XCM@definecolor{IndigoA200}{536DFE}
\XCM@definecolor{IndigoA400}{3D5AFE}
\XCM@definecolor{IndigoA700}{304FFE}
\definecolor{\XCM@prefix Indigo}{named}{\XCM@prefix Indigo500}
%    \end{macrocode}
%
% \subsection{Blue}
%    \begin{macrocode}
\XCM@definecolor{Blue50}  {E3F2FD}
\XCM@definecolor{Blue100} {BBDEFB}
\XCM@definecolor{Blue200} {90CAF9}
\XCM@definecolor{Blue300} {64B5F6}
\XCM@definecolor{Blue400} {42A5F5}
\XCM@definecolor{Blue500} {2196F3}
\XCM@definecolor{Blue600} {1E88E5}
\XCM@definecolor{Blue700} {1976D2}
\XCM@definecolor{Blue800} {1565C0}
\XCM@definecolor{Blue900} {0D47A1}
\XCM@definecolor{BlueA100}{82B1FF}
\XCM@definecolor{BlueA200}{448AFF}
\XCM@definecolor{BlueA400}{2979FF}
\XCM@definecolor{BlueA700}{2962FF}
\definecolor{\XCM@prefix Blue}{named}{\XCM@prefix Blue500}
%    \end{macrocode}
%
% \subsection{Light Blue}
%    \begin{macrocode}
\XCM@definecolor{LightBlue50}  {E1F5FE}
\XCM@definecolor{LightBlue100} {B3E5FC}
\XCM@definecolor{LightBlue200} {81D4FA}
\XCM@definecolor{LightBlue300} {4FC3F7}
\XCM@definecolor{LightBlue400} {29B6F6}
\XCM@definecolor{LightBlue500} {03A9F4}
\XCM@definecolor{LightBlue600} {039BE5}
\XCM@definecolor{LightBlue700} {0288D1}
\XCM@definecolor{LightBlue800} {0277BD}
\XCM@definecolor{LightBlue900} {01579B}
\XCM@definecolor{LightBlueA100}{80D8FF}
\XCM@definecolor{LightBlueA200}{40C4FF}
\XCM@definecolor{LightBlueA400}{00B0FF}
\XCM@definecolor{LightBlueA700}{0091EA}
\definecolor{\XCM@prefix LightBlue}{named}{\XCM@prefix LightBlue500}
%    \end{macrocode}
%
% \subsection{Cyan}
%    \begin{macrocode}
\XCM@definecolor{Cyan50}  {E0F7FA}
\XCM@definecolor{Cyan100} {B2EBF2}
\XCM@definecolor{Cyan200} {80DEEA}
\XCM@definecolor{Cyan300} {4DD0E1}
\XCM@definecolor{Cyan400} {26C6DA}
\XCM@definecolor{Cyan500} {00BCD4}
\XCM@definecolor{Cyan600} {00ACC1}
\XCM@definecolor{Cyan700} {0097A7}
\XCM@definecolor{Cyan800} {00838F}
\XCM@definecolor{Cyan900} {006064}
\XCM@definecolor{CyanA100}{84FFFF}
\XCM@definecolor{CyanA200}{18FFFF}
\XCM@definecolor{CyanA400}{00E5FF}
\XCM@definecolor{CyanA700}{00B8D4}
\definecolor{\XCM@prefix Cyan}{named}{\XCM@prefix Cyan500}
%    \end{macrocode}
%
% \subsection{Teal}
%    \begin{macrocode}
\XCM@definecolor{Teal50}  {E0F2F1}
\XCM@definecolor{Teal100} {B2DFDB}
\XCM@definecolor{Teal200} {80CBC4}
\XCM@definecolor{Teal300} {4DB6AC}
\XCM@definecolor{Teal400} {26A69A}
\XCM@definecolor{Teal500} {009688}
\XCM@definecolor{Teal600} {00897B}
\XCM@definecolor{Teal700} {00796B}
\XCM@definecolor{Teal800} {00695C}
\XCM@definecolor{Teal900} {004D40}
\XCM@definecolor{TealA100}{A7FFEB}
\XCM@definecolor{TealA200}{64FFDA}
\XCM@definecolor{TealA400}{1DE9B6}
\XCM@definecolor{TealA700}{00BFA5}
\definecolor{\XCM@prefix Teal}{named}{\XCM@prefix Teal500}
%    \end{macrocode}
%
% \subsection{Green}
%    \begin{macrocode}
\XCM@definecolor{Green50}  {E8F5E9}
\XCM@definecolor{Green100} {C8E6C9}
\XCM@definecolor{Green200} {A5D6A7}
\XCM@definecolor{Green300} {81C784}
\XCM@definecolor{Green400} {66BB6A}
\XCM@definecolor{Green500} {4CAF50}
\XCM@definecolor{Green600} {43A047}
\XCM@definecolor{Green700} {388E3C}
\XCM@definecolor{Green800} {2E7D32}
\XCM@definecolor{Green900} {1B5E20}
\XCM@definecolor{GreenA100}{B9F6CA}
\XCM@definecolor{GreenA200}{69F0AE}
\XCM@definecolor{GreenA400}{00E676}
\XCM@definecolor{GreenA700}{00C853}
\definecolor{\XCM@prefix Green}{named}{\XCM@prefix Green500}
%    \end{macrocode}
%
% \subsection{Light Green}
%    \begin{macrocode}
\XCM@definecolor{LightGreen50}  {F1F8E9}
\XCM@definecolor{LightGreen100} {DCEDC8}
\XCM@definecolor{LightGreen200} {C5E1A5}
\XCM@definecolor{LightGreen300} {AED581}
\XCM@definecolor{LightGreen400} {9CCC65}
\XCM@definecolor{LightGreen500} {8BC34A}
\XCM@definecolor{LightGreen600} {7CB342}
\XCM@definecolor{LightGreen700} {689F38}
\XCM@definecolor{LightGreen800} {558B2F}
\XCM@definecolor{LightGreen900} {33691E}
\XCM@definecolor{LightGreenA100}{CCFF90}
\XCM@definecolor{LightGreenA200}{B2FF59}
\XCM@definecolor{LightGreenA400}{76FF03}
\XCM@definecolor{LightGreenA700}{64DD17}
\definecolor{\XCM@prefix LightGreen}{named}{\XCM@prefix LightGreen500}
%    \end{macrocode}
%
% \subsection{Lime}
%    \begin{macrocode}
\XCM@definecolor{Lime50}  {F9FBE7}
\XCM@definecolor{Lime100} {F0F4C3}
\XCM@definecolor{Lime200} {E6EE9C}
\XCM@definecolor{Lime300} {DCE775}
\XCM@definecolor{Lime400} {D4E157}
\XCM@definecolor{Lime500} {CDDC39}
\XCM@definecolor{Lime600} {C0CA33}
\XCM@definecolor{Lime700} {AFB42B}
\XCM@definecolor{Lime800} {9E9D24}
\XCM@definecolor{Lime900} {827717}
\XCM@definecolor{LimeA100}{F4FF81}
\XCM@definecolor{LimeA200}{EEFF41}
\XCM@definecolor{LimeA400}{C6FF00}
\XCM@definecolor{LimeA700}{AEEA00}
\definecolor{\XCM@prefix Lime}{named}{\XCM@prefix Lime500}
%    \end{macrocode}
%
% \subsection{Yellow}
%    \begin{macrocode}
\XCM@definecolor{Yellow50}  {FFFDE7}
\XCM@definecolor{Yellow100} {FFF9C4}
\XCM@definecolor{Yellow200} {FFF59D}
\XCM@definecolor{Yellow300} {FFF176}
\XCM@definecolor{Yellow400} {FFEE58}
\XCM@definecolor{Yellow500} {FFEB3B}
\XCM@definecolor{Yellow600} {FDD835}
\XCM@definecolor{Yellow700} {FBC02D}
\XCM@definecolor{Yellow800} {F9A825}
\XCM@definecolor{Yellow900} {F57F17}
\XCM@definecolor{YellowA100}{FFFF8D}
\XCM@definecolor{YellowA200}{FFFF00}
\XCM@definecolor{YellowA400}{FFEA00}
\XCM@definecolor{YellowA700}{FFD600}
\definecolor{\XCM@prefix Yellow}{named}{\XCM@prefix Yellow500}
%    \end{macrocode}
%
% \subsection{Amber}
%    \begin{macrocode}
\XCM@definecolor{Amber50}  {FFF8E1}
\XCM@definecolor{Amber100} {FFECB3}
\XCM@definecolor{Amber200} {FFE082}
\XCM@definecolor{Amber300} {FFD54F}
\XCM@definecolor{Amber400} {FFCA28}
\XCM@definecolor{Amber500} {FFC107}
\XCM@definecolor{Amber600} {FFB300}
\XCM@definecolor{Amber700} {FFA000}
\XCM@definecolor{Amber800} {FF8F00}
\XCM@definecolor{Amber900} {FF6F00}
\XCM@definecolor{AmberA100}{FFE57F}
\XCM@definecolor{AmberA200}{FFD740}
\XCM@definecolor{AmberA400}{FFC400}
\XCM@definecolor{AmberA700}{FFAB00}
\definecolor{\XCM@prefix Amber}{named}{\XCM@prefix Amber500}
%    \end{macrocode}
%
% \subsection{Orange}
%    \begin{macrocode}
\XCM@definecolor{Orange50}  {FFF3E0}
\XCM@definecolor{Orange100} {FFE0B2}
\XCM@definecolor{Orange200} {FFCC80}
\XCM@definecolor{Orange300} {FFB74D}
\XCM@definecolor{Orange400} {FFA726}
\XCM@definecolor{Orange500} {FF9800}
\XCM@definecolor{Orange600} {FB8C00}
\XCM@definecolor{Orange700} {F57C00}
\XCM@definecolor{Orange800} {EF6C00}
\XCM@definecolor{Orange900} {E65100}
\XCM@definecolor{OrangeA100}{FFD180}
\XCM@definecolor{OrangeA200}{FFAB40}
\XCM@definecolor{OrangeA400}{FF9100}
\XCM@definecolor{OrangeA700}{FF6D00}
\definecolor{\XCM@prefix Orange}{named}{\XCM@prefix Orange500}
%    \end{macrocode}
%
% \subsection{Deep Orange}
%    \begin{macrocode}
\XCM@definecolor{DeepOrange50}  {FBE9E7}
\XCM@definecolor{DeepOrange100} {FFCCBC}
\XCM@definecolor{DeepOrange200} {FFAB91}
\XCM@definecolor{DeepOrange300} {FF8A65}
\XCM@definecolor{DeepOrange400} {FF7043}
\XCM@definecolor{DeepOrange500} {FF5722}
\XCM@definecolor{DeepOrange600} {F4511E}
\XCM@definecolor{DeepOrange700} {E64A19}
\XCM@definecolor{DeepOrange800} {D84315}
\XCM@definecolor{DeepOrange900} {BF360C}
\XCM@definecolor{DeepOrangeA100}{FF9E80}
\XCM@definecolor{DeepOrangeA200}{FF6E40}
\XCM@definecolor{DeepOrangeA400}{FF3D00}
\XCM@definecolor{DeepOrangeA700}{DD2C00}
\definecolor{\XCM@prefix DeepOrange}{named}{\XCM@prefix DeepOrange500}
%    \end{macrocode}
%
% \subsection{Brown}
%    \begin{macrocode}
\XCM@definecolor{Brown50} {EFEBE9}
\XCM@definecolor{Brown100}{D7CCC8}
\XCM@definecolor{Brown200}{BCAAA4}
\XCM@definecolor{Brown300}{A1887F}
\XCM@definecolor{Brown400}{8D6E63}
\XCM@definecolor{Brown500}{795548}
\XCM@definecolor{Brown600}{6D4C41}
\XCM@definecolor{Brown700}{5D4037}
\XCM@definecolor{Brown800}{4E342E}
\XCM@definecolor{Brown900}{3E2723}
\definecolor{\XCM@prefix Brown}{named}{\XCM@prefix Brown500}
%    \end{macrocode}
%
% \subsection{Grey}
%    \begin{macrocode}
\XCM@definecolor{Grey50} {FAFAFA}
\XCM@definecolor{Grey100}{F5F5F5}
\XCM@definecolor{Grey200}{EEEEEE}
\XCM@definecolor{Grey300}{E0E0E0}
\XCM@definecolor{Grey400}{BDBDBD}
\XCM@definecolor{Grey500}{9E9E9E}
\XCM@definecolor{Grey600}{757575}
\XCM@definecolor{Grey700}{616161}
\XCM@definecolor{Grey800}{424242}
\XCM@definecolor{Grey900}{212121}
\definecolor{\XCM@prefix Grey}{named}{\XCM@prefix Grey500}
%    \end{macrocode}
%
% \subsection{Blue Grey}
%    \begin{macrocode}
\XCM@definecolor{BlueGrey50} {ECEFF1}
\XCM@definecolor{BlueGrey100}{CFD8DC}
\XCM@definecolor{BlueGrey200}{B0BEC5}
\XCM@definecolor{BlueGrey300}{90A4AE}
\XCM@definecolor{BlueGrey400}{78909C}
\XCM@definecolor{BlueGrey500}{607D8B}
\XCM@definecolor{BlueGrey600}{546E7A}
\XCM@definecolor{BlueGrey700}{455A64}
\XCM@definecolor{BlueGrey800}{37474F}
\XCM@definecolor{BlueGrey900}{263238}
\definecolor{\XCM@prefix BlueGrey}{named}{\XCM@prefix BlueGrey500}
%    \end{macrocode}
%
% \subsection{Black \& White}
%    \begin{macrocode}
\XCM@definecolor{Black}{000000}
\XCM@definecolor{White}{FFFFFF}
%    \end{macrocode}
%
% \subsection[Google colors]{\google\ colors}
% The \google\ Indentity colors are available from 
% \url{https://design.google.com/articles/evolving-the-google-identity}.
%    \begin{macrocode}
\definecolor{GoogleBlue}  {HTML}{4285F4}
\definecolor{GoogleGreen} {HTML}{34A853}
\definecolor{GoogleYellow}{HTML}{FBBC05}
\definecolor{GoogleRed}   {HTML}{EA4335}
%    \end{macrocode}
%
% \section{User-level macros}
%
% \begin{macro}{\XCM@extractcolor}
% An internal macro to extract and convert color values.
%    \begin{macrocode}
\newcommand{\XCM@extractcolor}[2]{%
  \extractcolorspecs{#1}{\model}{\colorval}%
  \convertcolorspec{\model}{\colorval}{#2}\extractedcolor
  \extractedcolor
}
%    \end{macrocode}
% \end{macro}
% \begin{macro}{\printcolorvalue}
% A user level macro to provide color values formatted in three different color
% models: \textbf{HTML}, \textbf{RGB}, and \textbf{CMYK}. The default value for 
% the color model is |HTML|, that provides color values in hexadecimal format.
%    \begin{macrocode}
\newcommand*{\printcolorvalue}[2][HTML]{%
  \edef\XCM@requiredModel{#1}%
  \def\XCM@HTMLModel{HTML}%
  \def\XCM@RGBModel{RGB}%
  \def\XCM@CMYKModel{cmyk}%
  \ifx\XCM@requiredModel\XCM@HTMLModel
    \#\XCM@extractcolor{#2}{#1}%
  \else\ifx\XCM@requiredModel\XCM@RGBModel
    rgb(\XCM@extractcolor{#2}{#1})%
  \else\ifx\XCM@requiredModel\XCM@CMYKModel
    cmyk(\XCM@extractcolor{#2}{#1})%
  \else\fi\fi\fi
}
%    \end{macrocode}
% \end{macro}
% \begin{macro}{\colorsample}
% A user level macro for drawing color samples. It draws a |colorbox| with 
% the specified color and value. By default the color model is |HTML|, 
% the text color is |white|, and the colorbox width is |6.5em|.
%    \begin{macrocode}
\def\colorsample{%
  \@ifnextchar[%
    {\XCM@colorsample@i}
    {\XCM@colorsample@i[HTML]}%
}
\def\XCM@colorsample@i[#1]{%
  \@ifnextchar[%
    {\XCM@colorsample@ii{#1}}
    {\XCM@colorsample@ii{#1}[6.5em]}%
}
\def\XCM@colorsample@ii#1[#2]{%
  \@ifnextchar[%
    {\XCM@colorsample@iii{#1}{#2}}
    {\XCM@colorsample@iii{#1}{#2}[white]}%
}
\def\XCM@colorsample@iii#1#2[#3]#4{%
  \@ifnextchar[%
    {\XCM@colorsample@iv{#1}{#2}{#3}{#4}}
    {\XCM@colorsample@iv{#1}{#2}{#3}{#4}[#4]}%
}
\def\XCM@colorsample@iv#1#2#3#4[#5]{%
  \def\XCM@temp{#2}\ifx\XCM@temp\empty
    \def\XCM@boxwidth{6.5em}%
  \else
    \def\XCM@boxwidth{#2}%
  \fi
  \selectcolormodel{#1}
  \colorbox{#4}{%
    \hspace*{.3em}
    \parbox[b][9\baselineskip]{\XCM@boxwidth}{%
      \footnotesize\bfseries\sffamily\color{#3}
      \vspace*{.7em}
      \raggedright#5%
      \vfill
      \raggedright\texttt{\printcolorvalue[#1]{#4}}
      \vspace*{.7em}
    }
    \hspace*{.3em} 
  }
}
%    \end{macrocode}
% \end{macro}
% \xmpkg\ provides some macro definitions at begin document.
%    \begin{macrocode}
\AtBeginDocument{
%    \end{macrocode}
% There is a macro definition that is built on top of the \tikzpkg\ package. 
% This definition works if such package is loaded in the main document.
%    \begin{macrocode}
\@ifpackageloaded{tikz}{%
%    \end{macrocode}
% When \tikzpkg\ is loaded, a macro with key-value options is provided using
% the |\pgfkeys| tools. Those key-value macro arguments are optional, the 
% default value of each one is defined by the |/palette/both| style.
%    \begin{macrocode}
\pgfkeys{/palette/.is family, /palette,
  width/.estore in = \XCM@pltWidth,
  height/.estore in = \XCM@pltHeight,
  shape/.estore in = \XCM@pltShape,
  title/.estore in = \XCM@pltTitle,
  title text color/.estore in = \XCM@pltTTextColor,
  init text color/.estore in = \XCM@pltITextColor,
  altern text color/.estore in = \XCM@pltATextColor,
  variation font/.store in = \XCM@pltVFont,
  value font/.store in = \XCM@pltNFont,
  title font/.store in = \XCM@pltTFont,
  colorbox sep/.estore in = \XCM@pltColorboxSep,
  first colorbox sep/.estore in = \XCM@pltFirstCSep,
  title colorbox height/.estore in = \XCM@pltTitleHeight,
  primary palette/.estore in = \XCM@pltPrimary,
  secondary palette/.estore in = \XCM@pltSecondary,
  percent char/.estore in = \XCM@pltPercentSep,
  primary variation/.estore in = \XCM@pltPrimaryBase,
  primary toggle at/.estore in = \XCM@pltPToggle,
  secondary toggle at/.estore in = \XCM@pltSToggle,
  color model/.estore in = \XCM@ColorModel,
  both/.style = {width = .45\textwidth,
    height = 2\baselineskip,
    shape = rectangle,
    title text color = white,
    init text color = black,
    altern text color = white,
    colorbox sep = 0pt,
    first colorbox sep = .2\baselineskip,
    title colorbox height = 6\baselineskip,
    variation font = \sffamily,
    value font = \ttfamily,
    title font = \sffamily\bfseries,
    primary palette = {50,100,200,...,900},
    secondary palette = {A100,A200,A400,A700},
    percent char = {},
    primary variation = 500,
    primary toggle at = 1000,
    secondary toggle at = 1000,
    color model = HTML},
  primary/.style = {primary palette = {50,100,200,...,900},
    secondary palette={},
    primary variation = 500,
    percent char={}},
  basic/.style = {primary palette = {10,20,...,100},
    secondary palette={},
    primary variation = 100,
    percent char=!}
}
%    \end{macrocode}
% The |positioning| library of \tikzpkg\ is loaded to provide convenient node
% placement.
%    \begin{macrocode}
\usetikzlibrary{positioning}
%    \end{macrocode}
% \begin{macro}{\colorpalette}
% A macro for drawing color palettes using \tikzpkg\. It takes one mandatory 
% argument (\emph{base color name}) and optional key-value arguments.
%    \begin{macrocode}
\newcommand*{\colorpalette}[2][]{%
  \def\XCM@pltTitle{#2}%
  \def\XCM@colorname{#2}%
  \pgfkeys{/palette, both, #1}%
  \begin{tikzpicture}
    \selectcolormodel{\XCM@ColorModel}
    \tikzset{colorbox/.style={\XCM@pltShape, minimum width=\XCM@pltWidth,
      minimum height=\XCM@pltHeight, node distance=\XCM@pltColorboxSep,
      outer sep=0pt}}
    \tikzset{title/.style = {font=\XCM@pltTFont, color=\XCM@pltTTextColor,
      inner sep=1em}}
    \tikzset{variant/.style = {font=\XCM@pltVFont, inner sep=1em}}
    \tikzset{value/.style = {font=\XCM@pltNFont, inner sep=1em}}
    \node[colorbox, fill=\XCM@colorname\XCM@pltPercentSep\XCM@pltPrimaryBase,
      minimum height=\XCM@pltTitleHeight, alias=last] {};
    \node[title, anchor=north west, align=left] at (last.north west) 
      {\XCM@pltTitle};
    \node[title, anchor=south west] at (last.south west) {\XCM@pltPrimaryBase};
    \node[title, anchor=south east, align=right] at (last.south east) 
      {\texttt{\printcolorvalue[\XCM@ColorModel]{\XCM@colorname%
      \XCM@pltPercentSep\XCM@pltPrimaryBase}}};
    \def\auxcolor{\XCM@pltITextColor}%
    \def\first{1}%
    \foreach \variant [count=\i] in \XCM@pltPrimary {
      \ifx\i\first
        \node[node distance=\XCM@pltFirstCSep, inner sep=0pt, below=of last,
          alias=last] {};
      \else\fi
      \ifx\variant\XCM@pltPToggle
        \gdef\auxcolor{\XCM@pltATextColor}%
      \else\fi
      \node[colorbox, fill=\XCM@colorname\XCM@pltPercentSep\variant,
        below=of last, alias=last] {};
      \node[variant, anchor=west, color=\auxcolor] at (last.west) {\variant};
      \node[value, anchor=east, color=\auxcolor, align=right] at (last.east)
        {\printcolorvalue[\XCM@ColorModel]{\XCM@colorname%
        \XCM@pltPercentSep\variant}};
    }%
    \def\auxcolor{\XCM@pltITextColor}%
    \foreach \variant [count=\i] in \XCM@pltSecondary {
      \ifx\i\first
      \node[node distance=\XCM@pltFirstCSep, inner sep=0pt, below=of last,
        alias=last] {};
      \else\fi 
      \ifx\variant\XCM@pltSToggle
        \gdef\auxcolor{\XCM@pltATextColor}%
      \else\fi
      \node[colorbox, fill=\XCM@colorname\XCM@pltPercentSep\variant,
        below=of last, alias=last] {};
      \node[variant, anchor=west, color=\auxcolor] at (last.west) {\variant};
      \node[value, anchor=east, color=\auxcolor, align=right] at (last.east)
        {\printcolorvalue[\XCM@ColorModel]{\XCM@colorname%
        \XCM@pltPercentSep\variant}};
    }%
  \end{tikzpicture}%
}
%    \end{macrocode}
% \end{macro}
%    \begin{macrocode}
}{}
}
%    \end{macrocode}
% \Finale
\endinput